%%%%%%%%%%%%%%%%%%%%%%%%%%%%%%%%%%%%%%%%%
% Beamer Presentation
% LaTeX Template
% Version 1.0 (10/11/12)
%
% This template has been downloaded from:
% http://www.LaTeXTemplates.com
%
% License:
% CC BY-NC-SA 3.0 (http://creativecommons.org/licenses/by-nc-sa/3.0/)
%
%%%%%%%%%%%%%%%%%%%%%%%%%%%%%%%%%%%%%%%%%

%----------------------------------------------------------------------------------------
%	PACKAGES AND THEMES
%----------------------------------------------------------------------------------------

\documentclass{beamer}

\mode<presentation> {

% The Beamer class comes with a number of default slide themes
% which change the colors and layouts of slides. Below this is a list
% of all the themes, uncomment each in turn to see what they look like.

%\usetheme{default}
%\usetheme{AnnArbor}
%\usetheme{Antibes}
%\usetheme{Bergen}
%\usetheme{Berkeley}
%\usetheme{Berlin}
%\usetheme{Boadilla}
%\usetheme{CambridgeUS}
%\usetheme{Copenhagen}
%\usetheme{Darmstadt}
%\usetheme{Dresden}
%\usetheme{Frankfurt}
%\usetheme{Goettingen}
%\usetheme{Hannover}
%\usetheme{Ilmenau}
%\usetheme{JuanLesPins}
%\usetheme{Luebeck}
\usetheme{Madrid}
%\usetheme{Malmoe}
%\usetheme{Marburg}
%\usetheme{Montpellier}
%\usetheme{PaloAlto}
%\usetheme{Pittsburgh}
%\usetheme{Rochester}
%\usetheme{Singapore}
%\usetheme{Szeged}
%\usetheme{Warsaw}

% As well as themes, the Beamer class has a number of color themes
% for any slide theme. Uncomment each of these in turn to see how it
% changes the colors of your current slide theme.

%\usecolortheme{albatross}
%\usecolortheme{beaver}
%\usecolortheme{beetle}
%\usecolortheme{crane}
%\usecolortheme{dolphin}
%\usecolortheme{dove}
%\usecolortheme{fly}
%\usecolortheme{lily}
%\usecolortheme{orchid}
%\usecolortheme{rose}
%\usecolortheme{seagull}
%\usecolortheme{seahorse}
%\usecolortheme{whale}
%\usecolortheme{wolverine}

%\setbeamertemplate{footline} % To remove the footer line in all slides uncomment this line
%\setbeamertemplate{footline}[page number] % To replace the footer line in all slides with a simple slide count uncomment this line

%\setbeamertemplate{navigation symbols}{} % To remove the navigation symbols from the bottom of all slides uncomment this line

}

\usepackage[utf8]{inputenc}
\usepackage[russian]{babel}
\usepackage{graphicx} % Allows including images
\usepackage{booktabs} % Allows the use of \toprule, \midrule and \bottomrule in tables
\usepackage{hyperref}

%----------------------------------------------------------------------------------------
%	TITLE PAGE
%----------------------------------------------------------------------------------------

\title[]{Сборка словаря со-встречаемостей в bigARTM}

\author{Михаил Солоткий}
\institute[ВМК МГУ] % Your institution as it will appear on the bottom of every slide, may be shorthand to save space
{
\medskip
\textit{Московский государственный университет им. М.В.Ломоносова
\newline
Факультет вычислительной математики и кибернетики}
}
\date{\today} % Date, can be changed to a custom date

\begin{document}

\begin{frame}
\titlepage % Print the title page as the first slide
\end{frame}

\begin{frame}
\frametitle{Что хочется рассказать...}
\tableofcontents % Throughout your presentation, if you choose to use \section{} and \subsection{} commands, these will automatically be printed on this slide as an overview of your presentation
\end{frame}

%----------------------------------------------------------------------------------------
%	PRESENTATION SLIDES
%----------------------------------------------------------------------------------------

%------------------------------------------------
\section{О самой задаче} 
%------------------------------------------------

\subsection{Постановка задачи}

\begin{frame}
\frametitle{Постановка задачи}
\begin{itemize}
\item Имеется 2 текстовых файла: коллекция в формате vowpal wabbit и словарь валидных токенов в формате UCI vocab.
\item Параметры алгоритма: ширина окна, минимальная со-встречаемость документов и токенов.
\item На выходе получить 4 файла в одинаковом формате: в каждом записаны строки текста в виде: token\_id token\_id value
\item Обрабатывается только базовая модальность
\item Среди 4 файлов 2 файла со-встречаемостей: документной и абсолютной и 2 файла со значениями ppmi, посчитанными по этим со-встречаемостям
\end{itemize}
\end{frame}

%------------------------------------------------

\subsection{Ширина окна}

\begin{frame}
\frametitle{Ширина окна}
\begin{itemize}
\item Окно шириной k внутри документа - это k токенов слева от данного и k токенов справа. Если данный токен находится слишком близко к началу или концу документа, то берутся токены вплоть до границы документа.
\item Пример при k = 2:
\newline
\newline
Шла \color{green} Саша по \color{red} шоссе \color{green} и сосала \color{black} сушку
\newline
\newline
|@verb Шла |@default\_class Саша \color{green} по шоссе \color{red} и \color{black} |@verb сосала |@default\_class \color{green} сушку
\end{itemize}
\end{frame}

%------------------------------------------------

\subsection{Подробнее о выходных данных}

\begin{frame}
\frametitle{Подробнее о выходных данных}
\begin{itemize}
\item Побочный эффект моего алгоритма: все данные в выходных файлах отсортированы.
\item В файлах со-встречаемостей есть симметричные пары, в файлах ppmi - нет.
\end{itemize}
\end{frame}

%------------------------------------------------

\subsection{Документная и абсолютная со-встречаемости}

\begin{frame}
\frametitle{Документная и абсолютная со-встречаемости}
\begin{itemize}
\item Абсолютная со-встречаемость - количество раз, сколько данная пара встретилась в коллекции в окне заданной ширины - не количество окон!
\item Документная со-встречаемость - количество документов, в которых хотя бы раз встретилась данная пара в окне заданной ширины.
\end{itemize}
\end{frame}

%------------------------------------------------

\subsection{Как считались ppmi}

\begin{frame}
\frametitle{Как считались ppmi}
\begin{itemize}
\item
$ ppmi(u, v) = \left [log{\dfrac{p(u, v)}{p(u)p(v)}} \right]_{+} = 
\left [log \dfrac{n_{uv}n}{n_{u}n_{v}} \right]_{+} $
\item Для абсолютной со-встречаемости n - количество всевозможных пар в тексте, которые находятся в окне заданной ширины.
$$ n_{u} = \sum_{v}^{}n_{uv} $$
\item Для документной со-встречаемости n - число документов в коллекции
\item $ n_{u} $ - количество документов, в которых встретилась данная пара
\end{itemize}
\end{frame}

%------------------------------------------------
\section{Как запускать}

\begin{frame}
\frametitle{Как запускать}
\begin{itemize}
\item bigartm -c vw -v vocab --cooc-min-tf 200 --cooc-min-df 200 --cooc-window 10 --write-cooc-tf cooc\_tf --write-cooc-df cooc\_df --write-ppmi-tf ppmi\_tf --write-ppmi-df ppmi\_df
\item Пометки о запуске добавлены в BigARTM Command Line Utility и Python Guide » Tokens Co-occurrence and Coherence Computation
\end{itemize}
\end{frame}

%------------------------------------------------
\section{Обзор алгоритма}
%------------------------------------------------

\subsection{Шаг 1: чтение и составление батчей со-встречаемостей}

\begin{frame}
\frametitle{Шаг 1: чтение и составление батчей со-встречаемостей}
\begin{itemize}
\item Загрузка токенов из vocab в хэш-таблицу. Здесь каждому токену присваивается порядковый номер начиная с 1.
\item Документы коллекции читаюся некоторыми порциями в оперативную память, парсятся на токены.
\item Происходит проход по токенам, берутся пары, попавшие в окно и которые есть в vocab.
\item Факт их со-стречаемости запоминается, то есть пара "записывается" в красно-черное.
\item Содержимое дерева записывается в выходной файл в так называемый "Батч со-встречаемостей"
\item Берётся новая порция документов, и всё это многопоточно.
\end{itemize}
\end{frame}

%----------------------------------------------------------------------------------------

\subsection{Шаг 2: агрегация по батчам}

\begin{frame}
\frametitle{Шаг 2: агрегация по батчам}
\begin{itemize}
\item Открываются все или, если их слишком много, почти все батчи, из них читаются ячейки, которые в каждом батче отсортированы по номеру первого токена.
\item На ячейках создаётся приоритетная очередь.
\item Ячейка сливается с промежуточным буфером, а из промежуточного буфера готовые данные могут быть записаны в оканчательный выходной файл.
\end{itemize}
\end{frame}

%----------------------------------------------------------------------------------------

\subsection{Шаг 3: подсчёт ppmi}

\begin{frame}
\frametitle{Шаг 3: подсчёт ppmi}
\begin{itemize}
\item Читаются выходные файлы со-встречаемостей, по ним считаются значения ppmi
\end{itemize}
\end{frame}

%----------------------------------------------------------------------------------------

\end{document}
