%-------------------------
% Resume in Latex
% Author : Sourabh Bajaj
% License : MIT
%------------------------

\documentclass[letterpaper,11pt]{article}

\usepackage{latexsym}
\usepackage[empty]{fullpage}
\usepackage{titlesec}
\usepackage{marvosym}
\usepackage[usenames,dvipsnames]{color}
\usepackage{verbatim}
\usepackage{enumitem}
\usepackage[pdftex]{hyperref}
\usepackage{fancyhdr}
\usepackage{fontawesome}
\usepackage[utf8]{inputenc}
\usepackage[russian]{babel}

\pagestyle{fancy}
\fancyhf{} % clear all header and footer fields
\fancyfoot{}
\renewcommand{\headrulewidth}{0pt}
\renewcommand{\footrulewidth}{0pt}

% Adjust margins
\addtolength{\oddsidemargin}{-0.375in}
\addtolength{\evensidemargin}{-0.375in}
\addtolength{\textwidth}{1in}
\addtolength{\topmargin}{-.5in}
\addtolength{\textheight}{1.0in}

\urlstyle{same}

\raggedbottom
\raggedright
\setlength{\tabcolsep}{0in}

% Sections formatting
\titleformat{\section}{
  \vspace{-4pt}\scshape\raggedright\large
}{}{0em}{}[\color{black}\titlerule \vspace{-5pt}]

%-------------------------
% Custom commands
\newcommand{\resumeItem}[2]{
  \item\small{
    \textbf{#1}{: #2 \vspace{-2pt}}
  }
}

\newcommand{\resumeSubheading}[4]{
  \vspace{-1pt}\item
    \begin{tabular*}{0.97\textwidth}{l@{\extracolsep{\fill}}r}
      \textbf{#1} & #2 \\
      \textit{\small#3} & \textit{\small #4} \\
    \end{tabular*}\vspace{-5pt}
}

\newcommand{\resumeSubItem}[2]{\resumeItem{#1}{#2}\vspace{-4pt}}

\renewcommand{\labelitemii}{$\circ$}

\newcommand{\resumeSubHeadingListStart}{\begin{itemize}[leftmargin=*]}
\newcommand{\resumeSubHeadingListEnd}{\end{itemize}}
\newcommand{\resumeItemListStart}{\begin{itemize}}
\newcommand{\resumeItemListEnd}{\end{itemize}\vspace{-5pt}}
\newcommand{\RomanNumeralCaps}[1]{\MakeUppercase{\romannumeral #1}}

%-------------------------------------------
%%%%%%  CV STARTS HERE  %%%%%%%%%%%%%%%%%%%%


\begin{document}

%----------HEADING-----------------
\begin{tabular*}{\textwidth}{l @{\extracolsep{\fill}} c @{\extracolsep{\fill}} r}
  \faEnvelope \enspace mihlushi@yandex.ru & \textbf{\Large Михаил Солоткий (не готово пока) \hspace{30pt}} & Москва, Россия \\
  \faGithub \enspace \href{https://github.com/MichaelSolotky}{\color{blue} https://github.com/MichaelSolotky} && \faMobilePhone \enspace +7 967 291-08-66 \\
\end{tabular*}

\begin{tabular*}{\textwidth}{l @{\extracolsep{\fill}} c @{\extracolsep{\fill}} r}
  \faLinkedin \enspace \href{https://www.linkedin.com/in/michael-solotky/}{\color{blue} https://www.linkedin.com/in/michael-solotky/} \\
\end{tabular*}


%-----------EDUCATION-----------------
\section{Образование}
  \resumeSubHeadingListStart
      \item{
		Магистрант, специальность: прикладная математика и информатика \\        
        \textbf{\href{https://www.hse.ru/}{\color{blue} НИУ Высшая Школа Экономики}}
        \hfill
        Сентябрь 2019 -- Июнь 2021 \\
        \href{https://cs.hse.ru/}{\color{blue} Факультет Компьютерных Наук}, совместная программа со \\ \href{https://yandexdataschool.ru/}{\color{blue} Школой Анализа Данных Яндекса}
      }
  \resumeSubHeadingListEnd

  \resumeSubHeadingListStart
      \item{
        Бакалавр, специальность: прикладная математика и информатика, ср. балл 4.85 / 5.0 \\
        \textbf{\href{https://www.msu.ru/}{\color{blue} МГУ имени М.В. Ломоносова}}
        \hfill
        Сентябрь 2015 -- Июнь 2019 \\
        \href{https://www.msu.ru/info/struct/dep/vmc.html}{\color{blue} Факультет Вычислительной Математики и Кибернетики}
      }
  \resumeSubHeadingListEnd
{Планирую продолжать учёбу в аспирантуре с получением степени PhD}


%-----------EXPERIENCE-----------------
\section{Опыт работы}
  \resumeSubHeadingListStart

      \item{
        \textbf{Отдел машинного перевода. Яндекс}
        \hfill
        Июнь 2019 -- Сентябрь 2019 \\
        Стажёр-разработчик машинного обучения
      }
      \begin{itemize}
        \item .
      \end{itemize}

      \item{
        \textbf{Группа разработки голосовых технологий. Яндекс}
        \hfill
        Июнь 2018 -- Октябрь 2018 \\
        Стажёр-разработчик бэкэнда
      }
      \begin{itemize}
        \item Реализовано несколько методов сглаживания вероятностей в языковых моделях для распознавания речи
        \item Проведены эксперименты по сравнению качества для нахождения лучшей модели среди использованных
        \item Реализован оптимальный алгоритм построения n-граммных языковых моделей на C++ с использованием MapReduce, \textbf{время работы которого минимум в 3 раза меньше базовой реализации, а также он немного превосходит по метрике качества базовую реализацию на некоторых датасетах}
        \item Написан фреймворк с операциями, доступными из терминала
      \end{itemize}

  \resumeSubHeadingListEnd

%-----------PROJECTS-----------------
\section{Проекты}
\resumeSubHeadingListStart
    \item{
      \textbf{{BigARTM}{ (C++ Boost/STL, Protobuf, Travis, AppVeyor)}}
      \hfill
      Январь 2017 -- настоящее время
    } \\
    Библиотека с открытым кодом по тематическому моделированию, поддерживающая одновременное использование множества регуляризаторов \\
    \faGithub \enspace \href{https://github.com/bigartm/bigartm}{\color{blue} github.com/bigartm/bigartm}
    \begin{itemize}
      \item Разработан и реализован алгоритм параллельного сбора статистики со-встречаемостей пар слов, positive PMI на корпусах неограниченного размера \\
      \textbf{Обработка полного текста англоязычной Википедии на 8 ядерном процессоре intel core i5 8th gen за 6 часов}
      \item Отвечаю за парсинг входных данных
    \end{itemize}

  \resumeSubHeadingListEnd


%-----------OTHER EXPERIENCE-----------------
\section{Другой опыт}
  \resumeSubHeadingListStart
      \item{
        \textbf{{Тест простоты чисел}{ (C++) }} \\
        Реализация теста Миллера для детерминированной проверки больших чисел на простоту в рамках курса \href{https://www.kaspersky.com/}{\color{blue} Лаборатории Касперского} <<C++ и проблемы безопасности>> \\
        \textbf{Время работы на простых числах длины 100 примерно 4 секунды} \\
        \faGithub \enspace \href{https://github.com/MichaelSolotky/sandbox/tree/master/Cpp_old_tasks/Primality_tests}{\color{blue} github.com/MichaelSolotky/sandbox/tree/master/Cpp\_old\_tasks/Primality\_tests}
      }

      \item{
        \textbf{Машинное обучение (NumPy, Scipy)} \\
        Реализация различных алгоритмов машинного обучения с нуля \\
        \faGithub \enspace \href{https://github.com/MichaelSolotky/sandbox/tree/master/ML}{\color{blue} github.com/MichaelSolotky/sandbox/tree/master/ML}
      }
      \resumeSubHeadingListEnd


%--------TECHNICAL SKILLS------------
\section{Технические навыки}
  \resumeSubHeadingListStart
    \resumeSubItem{Языки. Использовались в работе}{C++, Python, C, Bash}
    \resumeSubItem{Языки. Базовые знания}{SQL, Assembly language}
    \resumeSubItem{Технологии}{MapReduce, Protobuf, C++ Boost, CMake, Make, SciPy, Scikit-learn, NumPy, Pandas}
    \resumeSubItem{Библиотеки глубокого обучения}{PyTorch, TensorFlow}
    \resumeSubItem{Инструменты}{Git, Subversion, UNIX/Linux, Travis, AppVeyor, \LaTeX}
  \resumeSubHeadingListEnd

%-------------------------------------------
\end{document}
